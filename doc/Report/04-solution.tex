\section{Solution} 
\label{sec:sol}
% (The actual solution in details. Note that there is no need for code or
% specific software component tools description here. Also, you do not explain things
% already known by the theory, e.g., do not start elaborating on what is clustering and
% how useful it is.)
In this section, we present an algorithm to deal with the problem. 
Furthermore, we also describe some optimizations, whose improvement will be 
shown in Section~\ref{sec:eval} along with the evaluation results.

\subsection{Proposed solution}
The solution we propose is based on the A-Priori algorithm. 

The general idea is that we first group the tweets by the period they belong to, 
and then, for each period, we find the popular topics in the period by
exploiting the A-Priori algorithm. In this case, the transactions are represented by 
the tweets and the items of a transaction are the relative tokens. Consequently, 
the frequent itemsets returned by the algorithm are the popular topics of the period.
So, we extract the popular topics in the period using the A-Priori algorithm, 
setting as threshold $s$. 

By doing so, we build another dataset which will be the input for another 
instance of the A-Priori: in 
the new dataset, we have a transaction for each period of time and the items of a transaction
are the popular topics in that period. As threshold for this second instance we use $r$.

This algorithm answer exactly the problem stated in Section~\ref{sec:ps}, as 
it finds the topics which are $s$-popular in a fraction of periods of time greater than or equal to $r$,
which is the definition of $s$-popular-$r$-consistent. 

Actually, since in the second instance of the algorithm we only need to find 
itemsets of size $1$, i.e. single topics,
we only need to count in how many periods of time a topic is found to be frequent.
Thus, we only need to perform the first pass of A-Priori, without the need to build 
candidate pairs and so on. The complete process is shown in Algorithm~\ref{alg:pcis}. 

\SetStartEndCondition{ (}{)}{)}\SetAlgoBlockMarkers{}{\}}%
% \SetKwProg{Fn}{dsadsadas}{dsadad}
% \SetKwFunction{FRecurs}{generate\_multilayer}%
\SetKwFor{For}{for }{ \textbf{do}}{}{}%
\SetKwFor{ForEach}{foreach }{ \textbf{do}}{}{}%
\SetKwFor{If}{if }{ \textbf{then}}{}{}%
\SetKwFor{While}{while }{ \textbf{do}}{}{}%
% \SetKwIF{If}{ElseIf}{Else}{if}{\{}{elif}{else\{}{}%
% \SetKwFor{While}{while}{\{}{}%
% \SetKwRepeat{Repeat}{repeat\{}{until}%
% \AlgoDisplayBlockMarkers
% \SetAlgoNoLine%
\begin{algorithm}   
    \caption{Popular consistent topics}
    \label{alg:pcis}
% \Fn{\FRecurs{asd}}
    % {
    \newcommand{\var}[1]{\relax\ifmmode\mathit{#1}\else $\mathit{#1}$\fi}
    \newcommand{\call}[1]{\relax\ifmmode\mathit{#1}\else $\mathit{#1}$\fi}
    
    \SetKwInOut{Input}{Input}
    \SetKwInOut{Output}{Output}
    \SetStartEndCondition{ }{}{}%
    \DontPrintSemicolon
    % \Fn{\FMain{$T$, $s$, $r$, $start$, $end$, $a$}}{
    \Input{
        % s, a, start, end, frequent_itemset_f=apriori
        \textsc{Tweet} \var{T} \abracks \\
        \tfloat\ \var{s} \\
        \tfloat\ \var{r} \\
        \textsc{Timestamp} \var{start} \\
        \textsc{Timestamp} \var{end} \\
        \tint\ \var{a} \\
    }
    
    $\var{groups} \gets groupByTime(\var{T}, \var{start}, \var{end}, \var{a})$
    

    $\var{counter} \gets $\textsc{HashMap}()

    \ForEach {$\var{group} \in \var{groups}$}{
        $\var{frequentTopics} \gets$ \call{APriori}(\var{group}, $\var{s}$)

        \ForEach{$\var{topic} \in \var{frequentTopics}$}{
            $\var{counter}[\var{topic}] \gets \var{counter}[\var{topic}] + 1$
        }
    }

    $\var{ret} \gets \emptyset$

    \ForEach{$\var{topic}, \var{count} \in \var{counter}$}{

        \If{$\var{count} \geq \var{r} * |\var{groups}|$}{

            $\var{ret} \gets \var{ret} \cup \{\var{topic}\}$

        }
    }
            
    \Return{ret}
\end{algorithm}


The classical A-Priori algorithm is described in Section~\ref{sec:rw}.
However, if the dataset is not too big, we can exploit this 
fact to implement it more efficiently, by storing some additional 
data structures. We now proceed to describe these optimizations. 

% \input{algorithms/apriori_standard.tex}

\subsection{Optimizations}

The idea is illustrated in Algorithm~\ref{alg:apriori_opt}.

\SetStartEndCondition{ (}{)}{)}\SetAlgoBlockMarkers{}{\}}%
% \SetKwProg{Fn}{dsadsadas}{dsadad}
% \SetKwFunction{FRecurs}{generate\_multilayer}%
\SetKwFor{For}{for }{ \textbf{do}}{}{}%
\SetKwFor{ForEach}{foreach }{ \textbf{do}}{}{}%
\SetKwFor{If}{if }{ \textbf{then}}{}{}%
\SetKwFor{While}{while }{ \textbf{do}}{}{}%
% \SetKwIF{If}{ElseIf}{Else}{if}{\{}{elif}{else\{}{}%
% \SetKwFor{While}{while}{\{}{}%
% \SetKwRepeat{Repeat}{repeat\{}{until}%
% \AlgoDisplayBlockMarkers
% \SetAlgoNoLine%



\begin{algorithm}   
    \caption{Efficient A-Priori}
    \label{alg:apriori_opt}
% \Fn{\FRecurs{asd}}
    % {
    \newcommand{\var}[1]{\relax\ifmmode\mathit{#1}\else $\mathit{#1}$\fi}
    \newcommand{\call}[1]{\relax\ifmmode\mathit{#1}\else $\mathit{#1}$\fi}
    
    \SetKwInOut{Input}{Input}
    \SetKwInOut{Output}{Output}
    \SetStartEndCondition{ }{}{}%
    \DontPrintSemicolon
    % \Fn{\FMain{$T$, $s$, $r$, $start$, $end$, $a$}}{
    \Input{
        % s, a, start, end, frequent_itemset_f=apriori
        \textsc{Transaction} \var{T} \abracks \\
        \tfloat\ \var{s} \\
    }

    $\var{k} \gets 1$
    
    $\var{C_k}, \var{indexTransactions}, \var{indexItem} \gets getItems(T)$

    $\var{result} \gets \emptyset$
    
    \While{$C_k \neq \emptyset$}{
        
        $\var{L_k} \gets \emptyset$

        \ForEach{$\var{c} \in \var{C_k}$}{

            $\var{support} \gets \call{getSupport}(\var{c}, \var{indexTransactions})$

            \If{$\var{support} \geq \var{s * |T|}$}{

                $\var{L_k} \gets \var{L_k} \cup \{\var{c}\}$

                \If{$|\var{c}| > 1$}{

                    $\var{result} \gets \var{result} \cup \{\var{c}\}$
                
                }
            }
        }
        
        $\var{C_{k+1}} \gets \call{nextCandidates}(\var{L_k}, k+1)$

        $\var{k} \gets \var{k} + 1$
        
        
    }

    \Return{$\call{mapItems}(\var{result}, \var{indexItem})$}

\end{algorithm}

    
    


First, Procedure~\ref{alg:getItems} is called. The procedure scans all the transactions 
and the items in them, building three data structures:
\begin{description}
    \item[itemIndex, indexItem] used to map items to numbers and vice versa. In fact, 
        the algorithm finds frequent itemsets of numbers corresponding to items, to 
        be more efficient both in terms of time and memory. 
        Once the itemsets of numbers are found, they are converted back to sets of items.  
    \item[items] that is a set composed of the sets containing single items, i.e. the candidates 
        of size $1$. 
    \item[indexTransactions] an array where the $i$-th entry is the set of the transactions 
        containing the $i$-th item.
\end{description}

The rest of the algorithm is almost the same as the classical 
version, except for the computation of the support of an itemset
in Procedure~\ref{alg:getSupport}, 
which exploits the \emph{indexTransactions} structure to speed up the 
computation. In fact, having in main memory the set of transactions 
containing each item allows us to get the support of the 
itemset fast by computing the size of the intersection of the transactions set of
each item in the itemset.

Finally, once found the frequent itemsets are found, the algorithm maps back 
set of numbers to sets of items, as shown in Procedure~\ref{alg:mapItems}.

% \input{algorithms/apriori_opt.tex}

\SetStartEndCondition{ (}{)}{)}\SetAlgoBlockMarkers{}{\}}%
% \SetKwProg{Fn}{dsadsadas}{dsadad}
% \SetKwFunction{FRecurs}{generate\_multilayer}%
\SetKwFor{For}{for }{ \textbf{do}}{}{}%
\SetKwFor{ForEach}{foreach }{ \textbf{do}}{}{}%
\SetKwFor{If}{if }{ \textbf{then}}{}{}%
\SetKwFor{While}{while }{ \textbf{do}}{}{}%
\SetKwIF{If}{ElseIf}{Else}{if}{ then}{elif}{else}{}%
% \SetKwFor{While}{while}{\{}{}%
% \SetKwRepeat{Repeat}{repeat\{}{until}%
% \AlgoDisplayBlockMarkers
% \SetAlgoNoLine%
\SetAlgorithmName{Procedure}{}{}
\begin{algorithm}
    % \renewcommand{\thealgocf}{}
    \caption{getItems}
    \label{alg:getItems}

% \Fn{\FRecurs{asd}}
    % {
    \newcommand{\var}[1]{\relax\ifmmode\mathit{#1}\else $\mathit{#1}$\fi}
    \newcommand{\call}[1]{\relax\ifmmode\mathit{#1}\else $\mathit{#1}$\fi}
    
    \SetKwInOut{Input}{Input}
    \SetKwInOut{Output}{Output}
    \SetStartEndCondition{ }{}{}%
    \DontPrintSemicolon
    % \Fn{\FMain{$T$, $s$, $r$, $start$, $end$, $a$}}{
    \Input{
        % s, a, start, end, frequent_itemset_f=apriori
        \textsc{Transactions} \var{T} \abracks \\
    }

    $\var{n} \gets 0$

    $\var{items} \gets \emptyset$

    $\var{indexTransactions} \gets \abracks$

    $\var{indexItem} \gets \abracks$
    
    $\var{itemIndex} \gets $ \textsc{HashMap}()

    $\var{i} \gets 0$

    \ForEach{$\var{t} \in \var{T}$}{
        
        \ForEach{$\var{item} \in \var{t}$}{
            
            
            \uIf{$\var{item} \notin \var{itemIndex}$}{
                $\var{j} \gets \var{n}$
               
                $\var{n} \gets \var{n} + 1$

                $\var{items} \gets \var{items} \cup \{\{\var{j}\}\}$
                
                $\var{itemIndex}[\var{item}] \gets \var{j}$
                
                $\var{indexItem}.\call{append}(\var{item})$

                $\var{indexTransactions}.\call{append}(\emptyset)$
            }\uElse{

                $\var{j} \gets \var{itemIndex}[\var{item}]$

            }

            $\var{indexTransactions}[\var{j}] \gets \var{indexTransactions}[\var{j}] \cup \{i\}$
            
            $\var{i} \gets \var{i} + 1$

        }
    }
    
    % items = [frozenset((i,)) for i in range(n_items)]
    \Return{$\var{items}, \var{indexTransactions}, \var{indexItem}$}
\end{algorithm}

\SetStartEndCondition{ (}{)}{)}\SetAlgoBlockMarkers{}{\}}%
% \SetKwProg{Fn}{dsadsadas}{dsadad}
% \SetKwFunction{FRecurs}{generate\_multilayer}%
\SetKwFor{For}{for }{ \textbf{do}}{}{}%
\SetKwFor{ForEach}{foreach }{ \textbf{do}}{}{}%
\SetKwFor{If}{if }{ \textbf{ then}}{}{}%
\SetKwFor{While}{while }{ \textbf{do}}{}{}%
\SetKwIF{If}{ElseIf}{Else}{if}{ then}{else if}{else}{}%
% \SetKwFor{While}{while}{\{}{}%
% \SetKwRepeat{Repeat}{repeat\{}{until}%
% \AlgoDisplayBlockMarkers
% \SetAlgoNoLine%
\SetAlgorithmName{Procedure}{}{}
\begin{algorithm}
    % \renewcommand{\thealgocf}{}
    \caption{getSupport}
    \label{alg:getSupport}
% \Fn{\FRecurs{asd}}
    % {
    \newcommand{\var}[1]{\relax\ifmmode\mathit{#1}\else $\mathit{#1}$\fi}
    \newcommand{\call}[1]{\relax\ifmmode\mathit{#1}\else $\mathit{#1}$\fi}
    
    \SetKwInOut{Input}{Input}
    \SetKwInOut{Output}{Output}
    \SetStartEndCondition{ }{}{}%
    \DontPrintSemicolon
    % \Fn{\FMain{$T$, $s$, $r$, $start$, $end$, $a$}}{
    \Input{
        % s, a, start, end, frequent_itemset_f=apriori
        \textsc{Set} \var{itemset} \\
        \textsc{Set} \var{indexTransactions} \abracks \\
    }

    
    % $\var{itemsetTrans} \gets \emptyset$
    
    % \ForEach{$\var{item} \in \var{itemset}$}{
        
    %     $\var{itemTrans} \gets \var{indexTransactions}[\var{item}]$
        
    %     \uIf{$\var{itemsetTrans} = \emptyset$}{

    %         $\var{itemsetTrans} \gets \var{itemTrans}$
            
    %     } \uElse {
        
    %         $\var{itemsetTrans} \gets \var{itemsetTrans} \cap \var{itemTrans}$

    %     }
        
    %     \If{$\var{itemsetTrans} = \emptyset$}{

    %         \Return{0}
            
    %     }
    % }

    % \Return{$|\var{itemsetTrans}|$}


    $\var{itemsetTrans} \gets \bigcap\limits_{\var{i} \in \var{itemset}}{\var{indexTransactions}[\var{i}]}$


    \Return{$|\var{itemsetTrans}|$} 

\end{algorithm}

    
    



    \SetStartEndCondition{ (}{)}{)}\SetAlgoBlockMarkers{}{\}}%
    % \SetKwProg{Fn}{dsadsadas}{dsadad}
    % \SetKwFunction{FRecurs}{generate\_multilayer}%
    \SetKwFor{For}{for }{ \textbf{do}}{}{}%
    \SetKwFor{ForEach}{foreach }{ \textbf{do}}{}{}%
    \SetKwFor{If}{if }{ \textbf{then}}{}{}%
    \SetKwFor{While}{while }{ \textbf{do}}{}{}%
    % \SetKwIF{If}{ElseIf}{Else}{if}{\{}{elif}{else\{}{}%
    % \SetKwFor{While}{while}{\{}{}%
    % \SetKwRepeat{Repeat}{repeat\{}{until}%
    % \AlgoDisplayBlockMarkers
    % \SetAlgoNoLine%
    \SetAlgorithmName{Procedure}{}{}
    \begin{algorithm}
        % \renewcommand{\thealgocf}{}
        \caption{mapItems}
        \label{alg:mapItems}
    % \Fn{\FRecurs{asd}}
        % {
        \newcommand{\var}[1]{\relax\ifmmode\mathit{#1}\else $\mathit{#1}$\fi}
        \newcommand{\call}[1]{\relax\ifmmode\mathit{#1}\else $\mathit{#1}$\fi}
        
        \SetKwInOut{Input}{Input}
        \SetKwInOut{Output}{Output}
        \SetStartEndCondition{ }{}{}%
        \DontPrintSemicolon
        % \Fn{\FMain{$T$, $s$, $r$, $start$, $end$, $a$}}{
        \Input{
            % s, a, start, end, frequent_itemset_f=apriori
            \textsc{Set} \var{frequentItemsets} \abracks \\
            \textsc{Item} \var{indexItem} \\
        }
        
        $\var{result} \gets \emptyset$

        \ForEach{$\var{itemset} \in \var{frequentItemsets}$}{
            $\var{tmp} \gets \bigcup\limits_{i \in \var{itemset}}{\{\var{indexItem}[\var{i}]\}}$
            
            % \ForEach{$\var{\var{item} \in \var{itemset}}$}{

            %     $\var{tmp} \gets \var{tmp} \cup \{\var{indexItem}[\var{item}]\}$
            
            % }

            $\var{result} \gets \var{result} \cup \{\var{tmp}\}$
        }
        

        \Return{\var{result}}
        
\end{algorithm}
