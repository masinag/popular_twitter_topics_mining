\section{Implementation}
% (Description of what tools you have used to implement the solution
% you described above)
The dataset manipulation and the algorithm have been developed and 
tested using Python $3.8.5$,
exploiting some useful tools offered by the language and the available
libraries.

\subsection{Dataset manipulation}
As we explain in Section~\ref{sec:ds}, the original dataset was precomputed 
to extract from each tweet the relevant terms by which it is composed.
For reading and cleaning the dataset the following libraries were used:
\begin{description}
    \item[Pandas] to read and write the csv;
    \item[HTML] to unescape html special characters;
    \item[Regex] to manipulate the text;
    \item[NLTK] to remove stop words, lemmatize words and split the text into tokens;    
\end{description}

\subsection{Algorithm implementation}
The A-Priori algorithm was implemented from scratch in order to adapt it to the specific 
case and apply the optimizations discussed in Section~\ref{sec:sol}. So, 
the only external library used was \emph{Pandas}, which was used to read the 
csv and to group the tweets by period of time.

The complete code is available on Github.~\cite{code:github}

