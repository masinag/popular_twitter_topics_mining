\section{Introduction and Motivation}
\label{sec:intro}
Everyday, millions of people access online communities and social networks, producing 
a huge amount of content, that, if properly analyzed, can be exploited in many different ways.
In particular, in this work we focus on Twitter, a social network where the main 
content produced by users are tweets. Tweets are short texts, up to 280 characters,
that are usually written by people to share with their followers something they think to be 
relevant. Tweets can contain, for instance, funny jokes, special personal events or facts
and opinions about current news. The latter are particularly important,
because they reflect a part of the public opinion on recent events.
Therefore, they can be exploited, for example, 
by politicians to gain consensus or by social networks or other online platforms to 
recommend contents about the most trending topics to better engage their users.
By observing which concepts are frequently mentioned together, we could 
identify the spread of some fake news. Also, trending topics 
may be a good indicator of how a TV series has been welcomed by the public, and 
film companies can regulate their investments accordingly.

The purpose of this work is to identify which topics become popular many times 
over time.
% The purpose of this work is to identify
% an effective method to identify consistent topics.

We define a topic as a set of terms, that can be words, hashtags
or another person's username. 
One could think that a trending 
topic is a topic that appears many times in the tweets of dataset, as in a typical 
market-basket scenario, where we have many baskets, in this case represented by 
tweets, each of which contains a small set of items, in this case represented by terms,
and we are asked to identify the most frequent combinations of items bought together by
customers.
However, in the context of Twitter, tweets have also associated a timestamp, 
that corresponds to the date and time the tweet has been posted. 

Considering this, topics may be appear many times in some intervals of time, 
when they are trending, while being less frequent in others. 
Our objective is to identify popular consistent topics in time, i.e.\ topics that 
are frequent in many periods of time.
In the next sections we will define more formally the problem 
and show a way to approach it. This procedure will then be evaluated and results will 
be analyzed.



