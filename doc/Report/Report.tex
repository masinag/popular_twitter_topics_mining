\documentclass[sigconf,edbt]{acmart-edbt2021}

\def\BibTeX{{\rm B\kern-.05em{\sc i\kern-.025em b}\kern-.08em
    T\kern-.1667em\lower.7ex\hbox{E}\kern-.125emX}}

\usepackage{booktabs} % For formal tables


% Copyright
\setcopyright{rightsretained}

% DOI
\acmDOI{}

% ISBN
\acmISBN{978-3-89318-084-4}

%Conference
\acmConference[EDBT 2021]{24th International Conference on Extending Database Technology (EDBT)}{March 23-26, 2021}{Nicosia, Cyprus} 
\acmYear{2021}

\settopmatter{printacmref=false, printccs=false, printfolios=false}

\pagestyle{empty} % removes running headers

\usepackage{numprint}
\npthousandsep{\,}
\usepackage[nodayofweek,level]{datetime}


\begin{document}
\title{Identify Twitter popular consistent topics in time}
% \titlenote{Produces the permission block, and copyright information}
% \subtitle{Extended Abstract}
% \subtitlenote{The full version of the author's guide is available as
%   \texttt{acmart.pdf} document}
  

\author{Gabriele Masina}
% \authornote{220496}
\orcid{220496}
\affiliation{%
  \institution{University of Trento,
  1\textsuperscript{st} year CS master's degree\\
  Matricola: 220496}
  \streetaddress{}
  \city{} 
  \state{} 
  \postcode{}
}
\email{gabriele.masina@studenti.unitn.it}

% \author{G.K.M. Tobin}
% \authornote{The secretary disavows any knowledge of this author's actions.}
% \affiliation{%
%   \institution{Institute for Clarity in Documentation}
%   \streetaddress{P.O. Box 1212}
%   \city{Dublin} 
%   \state{Ohio} 
%   \postcode{43017-6221}
% }
% \email{webmaster@marysville-ohio.com}

% \author{Lars Th{\o}rv{\"a}ld}
% \authornote{This author is the
%   one who did all the really hard work.}
% \affiliation{%
%   \institution{The Th{\o}rv{\"a}ld Group}
%   \streetaddress{1 Th{\o}rv{\"a}ld Circle}
%   \city{Hekla} 
%   \country{Iceland}}
% \email{larst@affiliation.org}

% \author{Valerie B\'eranger}
% \affiliation{%
%   \institution{Inria Paris-Rocquencourt}
%   \city{Rocquencourt}
%   \country{France}
% }
% \author{Aparna Patel} 
% \affiliation{%
%  \institution{Rajiv Gandhi University}
%  \streetaddress{Rono-Hills}
%  \city{Doimukh} 
%  \state{Arunachal Pradesh}
%  \country{India}}
% \author{Huifen Chan}
% \affiliation{%
%   \institution{Tsinghua University}
%   \streetaddress{30 Shuangqing Rd}
%   \city{Haidian Qu} 
%   \state{Beijing Shi}
%   \country{China}
% }

% \author{Charles Palmer}
% \affiliation{%
%   \institution{Palmer Research Laboratories}
%   \streetaddress{8600 Datapoint Drive}
%   \city{San Antonio}
%   \state{Texas} 
%   \postcode{78229}}
% \email{cpalmer@prl.com}

% \author{John Smith}
% \affiliation{\institution{The Th{\o}rv{\"a}ld Group}}
% \email{jsmith@affiliation.org}

% \author{Julius P.~Kumquat}
% \affiliation{\institution{The Kumquat Consortium}}
% \email{jpkumquat@consortium.net}

% The default list of authors is too long for headers}
% \renewcommand{\shortauthors}{B. Trovato et al.}
% \renewcommand{\shortauthors}{}


\begin{abstract}
Nowadays, social networks are observed with interest by companies and 
governments, since they reflect a part of the public opinion. In this paper 
we focus on Twitter trends, that are the most talked about topics on the 
social network in some period of time. 
We describe a method to identify popular consistent
topics over time in tweets. This method is then evaluated and compared
with some baseline method to identify its strengths and weaknesses.
\end{abstract}

%
% % The code below should be generated by the tool at
% % http://dl.acm.org/ccs.cfm
% % Please copy and paste the code instead of the example below. 
% %
% \begin{CCSXML}
% <ccs2012>
%  <concept>
%   <concept_id>10010520.10010553.10010562</concept_id>
%   <concept_desc>Computer systems organization~Embedded systems</concept_desc>
%   <concept_significance>500</concept_significance>
%  </concept>
%  <concept>
%   <concept_id>10010520.10010575.10010755</concept_id>
%   <concept_desc>Computer systems organization~Redundancy</concept_desc>
%   <concept_significance>300</concept_significance>
%  </concept>
%  <concept>
%   <concept_id>10010520.10010553.10010554</concept_id>
%   <concept_desc>Computer systems organization~Robotics</concept_desc>
%   <concept_significance>100</concept_significance>
%  </concept>
%  <concept>
%   <concept_id>10003033.10003083.10003095</concept_id>
%   <concept_desc>Networks~Network reliability</concept_desc>
%   <concept_significance>100</concept_significance>
%  </concept>
% </ccs2012>  
% \end{CCSXML}
% 
% \ccsdesc[500]{Computer systems organization~Embedded systems}
% \ccsdesc[300]{Computer systems organization~Redundancy}
% \ccsdesc{Computer systems organization~Robotics}
% \ccsdesc[100]{Networks~Network reliability}


% \keywords{ACM proceedings, \LaTeX, text tagging}

%% A "teaser" image appears between the author and affiliation
%% information and the body of the document, and typically spans the
%% page.
% \begin{teaserfigure}
%   \includegraphics[width=\textwidth]{sampleteaser}
%   \caption{Seattle Mariners at Spring Training, 2010.}
%   \label{fig:teaser}
% \end{teaserfigure}

\maketitle

\input{samplebody-conf}

%%
%% The next two lines define the bibliography style to be used, and
%% the bibliography file.
\bibliographystyle{ACM-Reference-Format}
\bibliography{sample-base}

%%
%% If your work has an appendix, this is the place to put it.
%% Please note that all the content must fit within the page limits, including any appendices.
%\appendix
%
%\section{Research Methods}
% ...

\end{document}
\endinput
