\section{Related Work}
\label{sec:rw}
% (max 1 page. You briefly describe for the methods you will use, what
% they do and what is their role. E.g. you describe what clustering does and what
% techniques exist for clustering.
In this chapter we introduce some concepts and techniques used to preprocess 
the dataset and to find popular consistent topics in the tweets.
We also briefly introduce what other methods could be used to find 
popular topics.

\mathchardef\mhyphen="2D
\subsection{TF-IDF}

In several applications, such as the one treated in this paper, 
it can be useful to categorize documents by their topic. Typically, this 
is done by finding, for each document, the words that characterize it.

Term Frequency - Inverse Document Frequency (TF-IDF) is a numerical statistic 
that is intended to reflect how important a word is for a document in a collection.~\cite{rw:tfidf}
This measure consists of a product of two components: TF and IDF.

\begin{definition}
    The \emph{Term Frequency} (\emph{TF}) of a word $i$ in a document $j$ is defined as
    \begin{equation*}
        \mathit{TF}_{ij} = \frac{\mathit{f}_{ij}}{\max_k{\mathit{f}_{kj}}}
    \end{equation*}
    where 
    $\mathit{f}_{ij}$ is the number of times $i$ appears in $j$.
\end{definition}

\begin{definition}
    The \emph{Inverse Document Frequency} (\emph{IDF}) of a word $i$ in a document $j$ is defined as
    \begin{equation*}
        \mathit{IDF}_{ij} = \log{\frac{N}{n_i}}
    \end{equation*}
    where $N$ is the total number of documents and $n_i$ is the number of documents the word $i$
    appears in.
\end{definition}

\begin{definition}
    The \emph{Term Frequency - Inverse Document Frequency} (\emph{TF-IDF}) of a word $i$ in a document 
    $j$ is defined as
    \begin{equation*}
        \mathit{TF \mhyphen IDF}_{ij} = \mathit{TF}_{ij} * \mathit{IDF}_{ij}
    \end{equation*}
\end{definition}

As we discuss in Section~\ref{sec:ds}, this technique can be used to preprocess the tweets in order to 
keep only relevant terms for each tweet. The objective is to obtain a more 
suitable dataset, excluding words that are used frequently but 
do not characterize the topic the tweets are about. We do this in order to obtain higher quality 
results and also to reduce the dimension of the input, so to speed up the execution time. 

\subsection{Finding Frequent Itemsets}
A subtask of the problem consists in finding popular topics in a period of time. This task can 
be reduced to the more general scenario in which we are given a list of
transactions, each containing a set of items. The problem to face is to find frequent 
itemsets, that are sets of items that appear 
together in many transactions, in order to find, for instance, association rules.
The fraction of transactions in which an itemset $i$ is present is called \emph{support} 
of $i$, and an itemset is said to be frequent if its support is at least $s$, where 
$s$ is a threshold set in advance.
An algorithm that solves this task efficiently is A-Priori.

\subsubsection{A-Priori}
The algorithm was first introduced in 1994 by Agrawal R.S. and Srikant R. and
it is a very fast algorithm to discover frequent itemsets in a
list of transactions.~\cite{rw:apriori}

The efficiency of the algorithm is based on the following observation:
if an itemset is frequent, then all its subsets are frequent too.
This property of monotonicity is exploited by the algorithm by 
seeing the implication in reverse way, that is itemsets whose 
subsets are not all frequent can not become frequent, so a great 
quantity of potential candidates is discarded.

The idea of A-Priori is shown in Algorithm~\ref{alg:apriori_basic}.
\SetStartEndCondition{ (}{)}{)}\SetAlgoBlockMarkers{}{\}}%
% \SetKwProg{Fn}{dsadsadas}{dsadad}
% \SetKwFunction{FRecurs}{generate\_multilayer}%
\SetKwFor{For}{for }{ \textbf{do}}{}{}%
\SetKwFor{ForEach}{foreach }{ \textbf{do}}{}{}%
\SetKwFor{If}{if }{ \textbf{then}}{}{}%
\SetKwFor{While}{while }{ \textbf{do}}{}{}%
% \SetKwIF{If}{ElseIf}{Else}{if}{\{}{elif}{else\{}{}%
% \SetKwFor{While}{while}{\{}{}%
% \SetKwRepeat{Repeat}{repeat\{}{until}%
% \AlgoDisplayBlockMarkers
% \SetAlgoNoLine%



\begin{algorithm}   
    \caption{A-Priori}
    \label{alg:apriori_basic}
% \Fn{\FRecurs{asd}}
    % {
    \newcommand{\var}[1]{\relax\ifmmode\mathit{#1}\else $\mathit{#1}$\fi}
    \newcommand{\call}[1]{\relax\ifmmode\mathit{#1}\else $\mathit{#1}$\fi}
    
    \SetKwInOut{Input}{Input}
    \SetKwInOut{Output}{Output}
    \SetStartEndCondition{ }{}{}%
    \DontPrintSemicolon
    % \Fn{\FMain{$T$, $s$, $r$, $start$, $end$, $a$}}{
    \Input{
        % s, a, start, end, frequent_itemset_f=apriori
        \textsc{Transaction} \var{T} \abracks \\
        \tfloat\ \var{s} \\
    }

    $\var{k} \gets 1$
    
    $\var{C_k} \gets items(T)$

    \While{$C_k \neq \emptyset$}{
        
        $\var{L_k} \gets $ itemsets in $C_k$ with a support $\geq s$
        
      
        $\var{C_{k+1}} \gets $ \parbox[t]{.6\linewidth}{itemsets of $k+1$ elements
        %  obtained by combining two members of $L_{k}$,
        having all subsets of size $k$ in $L_{k}$}
        
        $\var{k} \gets \var{k} + 1$
    }

    \Return{$\bigcup{\var{L_k}}$}

\end{algorithm}

    
    

At each iteration we have a set of candidate itemsets of size $k$ and 
we find among them frequent itemsets of size $k$. 
Initially, frequent itemsets of size $1$ are the sets containing a single item.
At each iteration, we get the frequent itemsets of size $k$, which are 
the itemsets among the candidates with a support at least $s$.
Then, we build the candidates of size $k+1$, by merging two frequent itemsets 
of size $k$ and checking if all the subsets of size $k$ of the new candidate 
are frequent. We repeat this process until the set of candidates is not empty.

Though A-Priori avoids counting many itemsets, there are numerous optimizations
that make a better use of the main memory.

Among these, PCY, the Multistage and the Multihash algorithms try to 
further reduce the memory used to count frequent itemsets
that are known an advance not to be candidates to 
become frequent.~\cite{alg:pcy}\cite{alg:multi} 

The SON algorithm applies a different heuristic, by dividing the file in 
chunks that can be stored in main memory to individuate
the candidate frequent itemsets more rapidly.~\cite{alg:son}

In this paper we focus just on the A-Priori algorithm, although the other
algorithms could be used to deal with larger datasets. 